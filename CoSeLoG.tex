\documentclass[]{article}
\usepackage[T1]{fontenc}
\usepackage{lmodern}
\usepackage{amssymb,amsmath}
\usepackage{ifxetex,ifluatex}
\usepackage{fixltx2e} % provides \textsubscript
% Set line spacing
% use upquote if available, for straight quotes in verbatim environments
\IfFileExists{upquote.sty}{\usepackage{upquote}}{}
\ifnum 0\ifxetex 1\fi\ifluatex 1\fi=0 % if pdftex
  \usepackage[utf8]{inputenc}
\else % if luatex or xelatex
  \ifxetex
    \usepackage{mathspec}
    \usepackage{xltxtra,xunicode}
  \else
    \usepackage{fontspec}
  \fi
  \defaultfontfeatures{Mapping=tex-text,Scale=MatchLowercase}
  \newcommand{\euro}{€}
\fi
% use microtype if available
\IfFileExists{microtype.sty}{\usepackage{microtype}}{}
\usepackage[margin=1in]{geometry}
\usepackage{graphicx}
% Redefine \includegraphics so that, unless explicit options are
% given, the image width will not exceed the width of the page.
% Images get their normal width if they fit onto the page, but
% are scaled down if they would overflow the margins.
\makeatletter
\def\ScaleIfNeeded{%
  \ifdim\Gin@nat@width>\linewidth
    \linewidth
  \else
    \Gin@nat@width
  \fi
}
\makeatother
\let\Oldincludegraphics\includegraphics
{%
 \catcode`\@=11\relax%
 \gdef\includegraphics{\@ifnextchar[{\Oldincludegraphics}{\Oldincludegraphics[width=\ScaleIfNeeded]}}%
}%
\ifxetex
  \usepackage[setpagesize=false, % page size defined by xetex
              unicode=false, % unicode breaks when used with xetex
              xetex]{hyperref}
\else
  \usepackage[unicode=true]{hyperref}
\fi
\hypersetup{breaklinks=true,
            bookmarks=true,
            pdfauthor={bdanalytics},
            pdftitle={Dutch Environmental Permit Application Process: CoSeLoG},
            colorlinks=true,
            citecolor=blue,
            urlcolor=blue,
            linkcolor=magenta,
            pdfborder={0 0 0}}
\urlstyle{same}  % don't use monospace font for urls
\setlength{\parindent}{0pt}
\setlength{\parskip}{6pt plus 2pt minus 1pt}
\setlength{\emergencystretch}{3em}  % prevent overfull lines
\setcounter{secnumdepth}{5}

%%% Change title format to be more compact
\usepackage{titling}
\setlength{\droptitle}{-2em}
  \title{Dutch Environmental Permit Application Process: CoSeLoG}
  \pretitle{\vspace{\droptitle}\centering\huge}
  \posttitle{\par}
  \author{bdanalytics}
  \preauthor{\centering\large\emph}
  \postauthor{\par}
  \date{}
  \predate{}\postdate{}




\begin{document}

\maketitle


{
\hypersetup{linkcolor=black}
\setcounter{tocdepth}{2}
\tableofcontents
}
\begin{center}\rule{0.5\linewidth}{\linethickness}\end{center}

\textbf{Date: (Thu) Dec 18, 2014}

\section{Background:}\label{background}

Data: Originates from the CoSeLoG project executed under NWO project
number 638.001.211. Within the CoSeLoG project the (dis)similarities
between several processes of different municipalities in the Netherlands
has been investigated. This event log contains the records of the
execution of the receiving phase of the building permit application
process in an anonymous municipality.

Source:
\url{http://data.3tu.nl/repository/uuid:a07386a5-7be3-4367-9535-70bc9e77dbe6}

Time period: 2010-10-02 to 2012-01-23

\section{Synopsis:}\label{synopsis}

Based on analysis utilizing process mining techniques, the CoSeLoG
process may be enhanced with the following recommendations:

\begin{enumerate}
\def\labelenumi{\arabic{enumi}.}
\item
  Normative Model Enhancements:

  1.1 Rename transitions / tasks / activities to highlight nature of
  activity rather than working on a ``receipt'' e.g.~TA (``Confirmation
  of receipt'') \& T01 through T05.

  1.2 Review need for many ``silent'' transitions \& places. For e.g.
  ``0 sink'', first ``tau split'' and the immediately succeeding places
  called ``source'' might not be really adding much value, especially if
  it's required more for process analytics rather than the actual
  business needs / process.

  1.3. Add guards to decision points utilizing case attributes which
  were either not available / utilized for this analysis.

  1.3. Analyze positive \& negative deviations in real-world cases
  utilizing decision points' guards. Negative deviations should create
  ``alerts'' to the appropriate personnel in real-time. Positive
  deviations should be utilized to enhance the normative model.
\item
  Bottleneck improvements:

  2.1 There are 791 cases that take 79.6 months to traverse from T05 to
  T06 which is approx. 31\% of the duration of all cases. However, this
  path is not allowed by the normative process model. Reducing this
  duration will have a big impact on the overall process duration.

  2.2 Similarly, there are 1,079 cases that take 29.6 months from TA to
  T02 which is also not allowed by the normative process model.

  2.3 To enhance the process and facilitate further analytics, capture
  transactional information about each task / acitivity (e.g.~start,
  complete, abort, schedule, assign, suspend, resume, withdraw, etc.)
\item
  Resource alloaction / utilization:

  3.1 Categorize resources into ``Generalists'' vs. ``Specialists'' to
  enhance both efficiency and effectiveness.

  3.2 Implement a Resource Recommendation System that considers resource
  type \& capacity along with case attributes \& time deadlines to
  recommend resources for the next event / activity for the case. Broad
  design guidelines may include:

  3.2.A Assign cases that require specialists based on case attributes /
  guards as they evolve in real-time. Once the case is assigned to a
  specialist, increase accountability by ensuring that case is processed
  all the way to completion by that specialist only, as much as
  possible.

  3.2.B If the case does not require a specialist, increase
  accountability by ensuring that case is processed all the way to
  completion by that generalist only, as much as possible.

  3.2.C Reducing the number of hand-offs (as long as that resource has
  capacity) will have a higher probability of identifying inherent
  bottlenecks and other efficiency improvements (e.g.~whenever a new
  resource has to step in for a case, that case needs to be ``learnt''
  by that resource prior to executing any task).
\end{enumerate}

\subsection{Potential next steps
include:}\label{potential-next-steps-include}

\begin{enumerate}
\def\labelenumi{\arabic{enumi}.}
\setcounter{enumi}{-1}
\item
  Change TA to T00 or T01 ?
\item
  Discover Petri net in ProM: 1.1 Add
  Fitness/Simplicity/Precision/Generalization to Petri net selection
  criteria. 1.1.1 Fitness metrics: alignment conformance of event log:
  ProM plug-in Replay a Log on Petri Net for Conformance Analysis w/
  option Measuring fitness

\begin{verbatim}
1.1.2 Simplicity metrics:
    # total places
    # silent places
    # silent transitions
    # arcs
    entropy: ProM plug-in ?
    minimal description length (MDL): ProM plug-in ?
1.1.3 Generalization metrics:
    alignment conformance of test event log (or cross-validation)

1.1.4 Precision metrics:   
    % model allowed traces (how do you deal with loops ?) present in event log: same as
    behavioral appropriateness ?: ProM plug-in Replay a Log on Petri Net for Conformance Analysis w/ option Measuring behavioral appropriateness
\end{verbatim}

  1.2 Add place names to discovered Petri nets. 1.3 Radar plot of model
  evaluations. 1.4 Display remaining tokens in selected Petri net.
\end{enumerate}

X.1 Review questions in Peer Assignment to ensure that all have been
answered\\X.2 Re-do listed steps to ensure reproduceability\\X.3 Convert
text to tables where appropriate\\X.4 Clean formats to make it more
readable X.5 Scan Peer Assignment Forum for enhancement ideas

\section{Import event log in Disco}\label{import-event-log-in-disco}

\textbf{Approach I used}:\\1. Import the event log into Disco.\\2.
Switch to ``Statistics'' tab / view\\3. Click on ``Overview'' button in
the left pane under ``Statistics views''\\4. Click on ``Events per
case'' button to the left of the graph

\textbf{What I saw}:

\includegraphics{CoSeLoG_Step_01_Q_01.png}

The graph pane displays a histogram (Number of cases) of Events per case
in this event log. The event log contains 8,577 events in 1,434 cases
with 27 activities.

\textbf{My analysis}:\\There are 6 events on average per case. This
information can be gathered by hovering the mouse on the tallest bar.

By clicking on ``Variants'' button on top of the table, we can see that
there are only 116 variants amongst the 1,434 cases.

The main observation from the `Events over time' graph is that the
maximum number of events (33) occured on May 2, 2011 across cases.

\section{Inspect process map in
Disco}\label{inspect-process-map-in-disco}

\textbf{Approach I used}:\\1. Click on ``Map'' tab in the window
header.\\2. Set ``Activities'' slider to 0\% \& ``Paths'' slider to 50\%
to make the process map fit on one screen and still be readable.

\textbf{What I saw}:

\includegraphics{CoSeLoG_Step_02.png}

\textbf{My analysis}:

The 6 most frequent activities between the initiation and termination of
cases in the process map include:\\A. Confirmation of receipt\\B. T02
Check confirmation of receipt\\C. T04 Determine confirmation of
receipt\\D. T05 Print and send confirmation of receipt\\E. T06 Determine
necessity of stop advice\\F. T10 Determine necessity to stop indication

The most frequent activity paths traced by the cases include (this is
supposed to display as a table, but doesn't work properly) :

\begin{verbatim}
                                    Activity Path | # of Cases  
\end{verbatim}

----------------------------------------------------- \textbar{}
-----------\\ Start -\textgreater{} TA -\textgreater{} End \textbar{}
116\\ Start -\textgreater{} TA -\textgreater{} T02 -\textgreater{} T04
-\textgreater{} T05 -\textgreater{} End \textbar{} 400\\Start
-\textgreater{} TA -\textgreater{} T02 -\textgreater{} T04
-\textgreater{} T05 -\textgreater{} T06 -\textgreater{} T10
-\textgreater{} End \textbar{} 828\\ \textbar{}\\ Total cases displayed
in this map \textbar{} 1,344\\ Total cases \textbar{} 1,434\\ \% cases
displayed in this map \textbar{} 94\%

There are 4 activities (TA, T02, T04 \& T05) regarding confirmation of
receipts. Maybe these activities are not named appropriately ?

\section{Inspect process performance in
Disco}\label{inspect-process-performance-in-disco}

\textbf{Approach I used}:\\1. Click on ``Performance'' bar / button in
the ``Detail'' pane (right above the ``Copy'' / ``Delete'' / ``Export''
icons).\\2. Select ``Total Duration'' in the ``Performance'' pane to
display.\\3. Select ``Case frequency'' as the secondary metric in the
``Performance'' pane to ensure that we don't use outliers (e.g.~low case
frequency) to make broad conclusions about the process.\\4. Cycle
through different metrics in the button next to ``Show:'' in the
Performance pane.

\textbf{What I saw}:

\includegraphics{CoSeLoG_Step_03.png}

The color \& thickness of the arcs are based on the distribution of the
selected primary performance metric. Additionally, if an arc is clicked,
a statistics window is displayed for that arc.

\textbf{My analysis}:\\\emph{Total Duration}: The arc from T05 to T06
takes 79.6 months for 791 cases (31\% of total duration of all cases
which is 258.12 months: mean of 5.4 days per case X 1,434 cases / 30
elapsed days per month). The next bottleneck seems to be TA
-\textgreater{} T02 which is 29.6 months for 1,079 cases.

Analysis of other metrics (median, mean \& max duration) highlighted
arcs with very low case frequency.

\section{Inspect event log in ProM}\label{inspect-event-log-in-prom}

\textbf{Approach I used}:

\begin{enumerate}
\def\labelenumi{\arabic{enumi}.}
\itemsep1pt\parskip0pt\parsep0pt
\item
  Click on ``import\ldots{}'' icon on the upper right hand side of the
  ``Workspace'' pane.\\
\item
  Click on eye icon (the one associated with the log in the middle; NOT
  the top one).\\
\item
  Click on ``Create new\ldots{}'' droplist in the top center of the
  window.\\
\item
  Select ``XDotted Chart'' by scrolling down the list.
\item
  Select ``Dotted Chart'' tab on the left.
\item
  Select ``Occurence of first event'' from the droplist for ``Case
  order:'' option.\\
\item
  Click on ``Apply Settings'' button.
\end{enumerate}

\textbf{What I saw}:

\includegraphics{CoSeLoG_Step_04.png}

Events for each case are plotted across time and color-coded. Did not
see the `size shows \# of events'-option. Zooming in does not make the
timeline any more readable / discernible (e.g.~do events initiate on
weekends ?)

\textbf{My analysis}:\\The arrival of the new cases is fairly constant
evidenced by the -45 degree slope of the (approx) line of blue dots.
There are some minor fluctuations which is difficult to quantify
(clicking on the dots does not display any additional information).

For the more recent cases there are a lot less events / activities
occuring close to case initiation compared to the earlier cases.

\section{Discover Petri net in ProM}\label{discover-petri-net-in-prom}

\textbf{Approach I used}:

\begin{enumerate}
\def\labelenumi{\arabic{enumi}.}
\itemsep1pt\parskip0pt\parsep0pt
\item
  Click on ``Actions'' icon.\\
\item
  Add imported event log to ``Input''.\\
\item
  Search for ``Alpha'' plug-in.\\
\item
  Select ``Mine for a Petri Net using Alpha-algorithm''.
\item
  Click on ``Start'' button.
\end{enumerate}

\textbf{What I saw}:

\includegraphics{CoSeLoG_Step_05.png}

This is clearly difficult to work with. Let's filter the event log to
make it more comprehensible.

\textbf{Approach I used}:

\begin{enumerate}
\def\labelenumi{\arabic{enumi}.}
\setcounter{enumi}{5}
\itemsep1pt\parskip0pt\parsep0pt
\item
  Click on ``Actions'' icon.\\
\item
  Search for ``Filter Log''.\\
\item
  Select ``Filter Log using Simple Heuristics''.\\
\item
  Click on ``Start'' button.
\item
  Change Log name to ``CoSeLoG (filtered on simple heuristics)''.\\
\item
  Click on ``Next'' button.
\item
  Select ``Select top percentage'' to 100\% because there is only 1
  Start event.
\item
  Click on ``Next'' button.
\item
  Select ``Select top percentage'' to 100\% because ideally keeping all
  End events would be critical in understanding the process.\\
\item
  Click on ``Next'' button.
\item
  Select ``Select top percentage'' to 96\% because this Event filter
  criterion discards many events and therefore many arcs in the
  resulting Petri net.
\item
  Change Log name to ``CoSeLoG (96\% filtered on simple heuristics)''.\\
\item
  Click on ``Finish'' button.
\end{enumerate}

\textbf{What I saw}:

\includegraphics{CoSeLoG_Step_05_Filter96.png}

The number of Event classes has gone down from 27 to 9. The number of
Events has reduced from 8,577 to 8,252 but number of Cases remain the
same.

\textbf{Approach I used}:

\begin{enumerate}
\def\labelenumi{\arabic{enumi}.}
\setcounter{enumi}{18}
\itemsep1pt\parskip0pt\parsep0pt
\item
  Click on ``Workspace'' icon.
\item
  Select ``CoSeLoG (filtered\ldots{}''.\\
\item
  Click on ``Actions'' icon.\\
\item
  Repeat tasks numbered 1-5 listed earlier in this Step. For task 2,
  select ``CoSeLoG (96\% filtered\ldots{})'' log to ``Input''.
\end{enumerate}

\textbf{What I saw}:

\includegraphics{CoSeLoG_Step_05_Filter96_PetriNet_Alpha.png}

The Alpha algorithm has discovered 9 transactions \& 9 places. However,
transactions T06, T07-1 \& T10 are not integrated well into the rest of
the control-flow.

\textbf{Approach I used}:

\begin{enumerate}
\def\labelenumi{\arabic{enumi}.}
\setcounter{enumi}{22}
\itemsep1pt\parskip0pt\parsep0pt
\item
  Click on ``Actions'' icon.\\
\item
  Add ``CoSeLoG (96\% filtered\ldots{})'' log to ``Input''.\\
\item
  Search for ``ILP'' plug-in.\\
\item
  Select ``Mine for a Petri Net using ILP''.
\item
  Click on ``Start'' button.
\item
  Select the ``Number of places'' option to ``Before \& After
  Transition'' instead of ``Per Causal Dependency'' to ensure clear
  ``End'' states \& minimize number of arcs.
\item
  Click ``Finish'' button.
\end{enumerate}

\textbf{What I saw}:

\includegraphics{CoSeLoG_Step_05_Filter96_PetriNet_ILP.png}

The ILP algorithm has discovered 9 transactions \& 8 places.
Additionally, the ILP Petri net handles transactions T06, T07-1 \& T10
better by not isolating them from the control-flow.

\textbf{Approach I used}:

\begin{enumerate}
\def\labelenumi{\arabic{enumi}.}
\setcounter{enumi}{29}
\itemsep1pt\parskip0pt\parsep0pt
\item
  Click on ``Actions'' icon.\\
\item
  Add ``CoSeLoG (96\% filtered\ldots{})'' log to ``Input''.\\
\item
  Search for ``Heuristics'' plug-in.\\
\item
  Select ``Mine for a Heuristics Net using Heuristics Miner''.
\item
  Click on ``Start'' button.
\item
  Select the default options and Click ``Continue'' button.
\item
  Click on ``Zoom'' button to the left of the graphic.\\
\item
  Select zoom level next to ``Fit \textgreater{}'' on the slider to view
  the net in its entirety.
\item
  Capture screen image.\\
\item
  Select zoom level to 50\% to make the net more readable.
\end{enumerate}

\textbf{What I saw}:

\includegraphics{CoSeLoG_Step_05_Filter96_Heuristics_Net.png}

9 transactions are discovered with a fitness score of 0.63 but T03,
T07-1 \& T11 are grayed out due to low case frequency (\textless{}= 55).

\textbf{Approach I used}:

\begin{enumerate}
\def\labelenumi{\arabic{enumi}.}
\setcounter{enumi}{39}
\itemsep1pt\parskip0pt\parsep0pt
\item
  Click on ``Workspace'' icon.\\
\item
  Select ``Mined Models'' of type ``HeuristicsNet''.\\
\item
  Click on ``Actions'' icon.\\
\item
  Select ``Convert Heuristics net into Petri net'' plug-in.\\
\item
  Click on ``Start'' button.
\end{enumerate}

\textbf{What I saw}:

\includegraphics{CoSeLoG_Step_05_Filter96_PetriNet_Heuristics.png}

This approach has discovered 9 transactions again, 14 ``hidden'' /
``silent'' transactions and 18 places. However, there does not seem to
be a clear ``End'' place.

\textbf{Approach I used}:

\begin{enumerate}
\def\labelenumi{\arabic{enumi}.}
\setcounter{enumi}{44}
\itemsep1pt\parskip0pt\parsep0pt
\item
  Click on ``Workspace'' icon.\\
\item
  Select ``CoSeLoG (96\% filtered\ldots{})'' log.\\
\item
  Click on ``Actions'' icon.\\
\item
  Search for ``Inductive'' plug-in.\\
\item
  Select ``Mine Petri net with Inductive Miner'' plug-in.\\
\item
  Click on ``Start'' button.
\item
  Change ``Variant'' option from default of ``Inductive Miner -
  infrequent'' to ``Inductive Miner'' because the default option drops
  T04 transaction probably due to infrequent cases containing it. We
  want to keep this transaction so that we can compare the different
  Petri nets with the same set of transactions.\\
\item
  Click ``Finish'' button.
\end{enumerate}

\textbf{What I saw}:

\includegraphics{CoSeLoG_Step_05_Filter96_PetriNet_Inductive.png}

This approach discovered 9 transactions, 25 ``hidden'' / ``silent''
transactions \& 21 places.

\textbf{My analysis}:

In my opinion, the ILP discovered Petri net is the best based on the
following criteria:

`+' Clear Start \& End states (ILP, Alpha \& Inductive; some combination
of options in the Heuristics plug-ins might generate a clear End state
too which I did not try due to too many steps).\\`+' Integrate all event
log transactions into the control-flow (ILP, Heuristics \&
Inductive).\\`+' No silent transactions (ILP \& Alpha).\\`+' Less arcs
(ILP \& Heuristics).

This should be displayed in a table for better readability ?

Since the analysis objective / goal is not known yet, these criteria
might be modified when that becomes clear.

\includegraphics{CoSeLoG_Step_05_Filter96_PetriNet_ILP.png}

The main traces include:\\\textbf{1}. Start -\textgreater{} TA
-\textgreater{} T02 -\textgreater{} T05 -\textgreater{} End01 with 2
tokens remaining in ILP3 (between TA \& T10) and ILP5 (between TA \&
T07-1)

\textbf{2}. Start -\textgreater{} TA -\textgreater{} T10 -\textgreater{}
T11 -\textgreater{} End02 with 1 token remaining in ILP1 (between TA \&
T02)

The traces with some loops include:\\\textbf{1A}. Start -\textgreater{}
TA -\textgreater{} T02 -\textgreater{} {[}T04{]}* -\textgreater{} T05
-\textgreater{} End01\\ After T02, there might be any number of T04
firings

\textbf{1B}. Start -\textgreater{} TA -\textgreater{} T02
-\textgreater{} {[}T03 -\textgreater{} T02{]}* -\textgreater{} T05
-\textgreater{} End01\\ After T02, there might be any number of T03
-\textgreater{} T02 loops

\textbf{1AB}. Start -\textgreater{} TA -\textgreater{} T02
-\textgreater{} {[}T04{]}* -\textgreater{} {[}T03 -\textgreater{}
T02{]}* -\textgreater{} T05 -\textgreater{} End01\\ After T02, there
might be any number of T04 firings and/or T03 -\textgreater{} T02 loops

\textbf{2A}. Start -\textgreater{} TA -\textgreater{} {[}T06{]}*
-\textgreater{} T10 -\textgreater{} T11 -\textgreater{} End02

\textbf{2B}. Start -\textgreater{} TA -\textgreater{} {[}T07-1{]}*
-\textgreater{} T10 -\textgreater{} T11 -\textgreater{} End02

\textbf{2AB}. Start -\textgreater{} TA -\textgreater{} {[}T06{]}*
-\textgreater{} {[}T07-1{]}* -\textgreater{} T10 -\textgreater{} T11
-\textgreater{} End02

\textbf{2C1a}. Start -\textgreater{} TA -\textgreater{} T10
-\textgreater{} T02 -\textgreater{} {[}T04{]}* -\textgreater{} {[}T03
-\textgreater{} T02{]}* -\textgreater{} T05 -\textgreater{}
End01\\\textbf{2C1b}. Start -\textgreater{} TA -\textgreater{} T10
-\textgreater{} T11 -\textgreater{} T02 -\textgreater{} {[}T04{]}*
-\textgreater{} {[}T03 -\textgreater{} T02{]}* -\textgreater{} T05
-\textgreater{} End01

\textbf{2AC1a}. Start -\textgreater{} TA -\textgreater{} {[}T06{]}*
-\textgreater{} T10 -\textgreater{} T02 -\textgreater{} {[}T04{]}*
-\textgreater{} {[}T03 -\textgreater{} T02{]}* -\textgreater{} T05
-\textgreater{} End01\\\textbf{2AC1b}. Start -\textgreater{} TA
-\textgreater{} {[}T06{]}* -\textgreater{} T10 -\textgreater{} T11
-\textgreater{} T02 -\textgreater{} {[}T04{]}* -\textgreater{} {[}T03
-\textgreater{} T02{]}* -\textgreater{} T05 -\textgreater{} End01

\textbf{2BC1a}. Start -\textgreater{} TA -\textgreater{} {[}T07-1{]}*
-\textgreater{} T10 -\textgreater{} T02 -\textgreater{} {[}T04{]}*
-\textgreater{} {[}T03 -\textgreater{} T02{]}* -\textgreater{} T05
-\textgreater{} End01\\\textbf{2BC1b}. Start -\textgreater{} TA
-\textgreater{} {[}T07-1{]}* -\textgreater{} T10 -\textgreater{} T11
-\textgreater{} T02 -\textgreater{} {[}T04{]}* -\textgreater{} {[}T03
-\textgreater{} T02{]}* -\textgreater{} T05 -\textgreater{} End01

\textbf{2ABC1a}. Start -\textgreater{} TA -\textgreater{} {[}T06{]}*
-\textgreater{} {[}T07-1{]}* -\textgreater{} T10 -\textgreater{} T02
-\textgreater{} {[}T04{]}* -\textgreater{} {[}T03 -\textgreater{}
T02{]}* -\textgreater{} T05 -\textgreater{} End01\\\textbf{2ABC1b}.
Start -\textgreater{} TA -\textgreater{} {[}T06{]}* -\textgreater{}
{[}T07-1{]}* -\textgreater{} T10 -\textgreater{} T11 -\textgreater{} T02
-\textgreater{} {[}T04{]}* -\textgreater{} {[}T03 -\textgreater{}
T02{]}* -\textgreater{} T05 -\textgreater{} End01

All the traces that end in End01 have 2 tokens remaining as described
for Trace 1.\\All the traces that end in End02 have 1 token remaining as
described for Trace 2.

These traces should be in a table for better comprehension ?

\section{Inspect conformance with normative model in
ProM}\label{inspect-conformance-with-normative-model-in-prom}

\textbf{Approach I used}:

\begin{enumerate}
\def\labelenumi{\arabic{enumi}.}
\itemsep1pt\parskip0pt\parsep0pt
\item
  Click on ``Workspace'' icon.\\
\item
  Click on ``import\ldots{}'' button.\\
\item
  Select the normative model file.\\
\item
  Select the `PNML Petri net files' importer.\\
\item
  Click on ``Actions'' icon.\\
\item
  Search for ``Replay'' plug-in.\\
\item
  Select `Replay a Log on Petri Net for Conformance Analysis' (not the
  variant with performance!) plug-in.
\item
  Add original event log to ``Input''.\\
\item
  Click on ``Start'' button.\\
\item
  Click `yes' in the `No Final Marking' pop-up.
\item
  Select the `sink' place on the left (note: do not select `0-sink'
  etc.) and click the button `Add Place \textgreater{}\textgreater{}' to
  add the place `sink' to the candidate final marking list.
\item
  Click `Finish' in the mapping wizard.\\
\item
  Click `Finish' .\\
\item
  Click `No, I've mapped all necessary event classes' to indicate that
  some events are not present in the normative model.\\
\item
  Click `Next'.\\
\item
  Click `Finish'.
\end{enumerate}

\textbf{What I saw}:

\includegraphics{CoSeLoG_Step_06_PetriNet_Normative_Conformance.png}

\emph{Transitions}: Most of the traces pass through very few ``labeled''
(tau are ``silent'') transitions: TA -\textgreater{} T06 -\textgreater{}
T10 \& TA -\textgreater{} T04 -\textgreater{} T05. The color darkness or
``fill'' of the transition boxes is based on the number of traces in the
event log that fire them. The numbers underneath the label in the
transition boxes refer to the number of synchronous moves vs. ``move on
model''. T13 \& T18 are never fired in this event log.

\emph{Places}: Place size displays ``move on log'' frequency. Places
where move log occured are colored yellow. However, size of ``source''
\& ``sink'' are not adjusted. Clicking on the place displays the
underlying label. Size of places going to silent transitions are not
adjusted with frequency but are colored yellow when there are move(s) on
log.

\emph{Arcs}: The thickness of the arcs seems proportional to the
frequency of event log traces.

\textbf{My analysis}:

The replay fitness (the `trace fitness' statistic) of the event log on
the normative process model is 0.8425. T10 has the maximum deviations
(151). T06 has the minimum (125).

The transition `T06 Determine necessity of stop advice+complete' (on the
top left of the model) was tested with 1,434 traces in the event log.
Out of those 1,309 (91\%) were synchronous moves in both the model \&
log. Amongst those 1,309 traces, T06 was fired synchronously for 1,327
times (i.e.~some traces fired T06 fired multiple times). For 125 traces,
T06 was fired in the model only.

\section{Inspect resource utilization in
ProM}\label{inspect-resource-utilization-in-prom}

\textbf{Approach I used}:\\1. Click on ``Workspace'' icon.\\2. Select
event log resource.\\3. Click on ``Actions'' icon for this resource.\\4.
Search for ``Mine for a Subcontracting Social Network'' \& select
it.\\5. Keep selected default options \& click on ``Continue''
button.\\6. Click ``Start'' button.\\7. Select the following in the
Social Network view options:\\ Layout: ISOMLayout\\ Ranking:
Betweenness\\ Mouse Mode: Picking\\ Edges removed for cl\ldots{}: 3rd
tick from left\\ View options:\\ shape by degree\\ show vertex names\\
show edge\\ Group Clusters\\8. Select a resource and move it to get more
clarity in the visual.

\textbf{What I saw}:

\includegraphics{CoSeLoG_Step_07_SocialNet.png}

Resources grouped by cluster, shaped by connectivity degree and edges
depicting the connectivity density.

\textbf{My analysis}:

The resources may be grouped into the following categories:\\I.
\emph{Singletons}: Resources \{31, 36 \& 42\} don't and \{16\} rarely
subcontract work.\\II. \emph{Couples}: \{05, 21\} and \{03, 14\} are
targets of subcontracting and amongst each other within the group. III.
\emph{General Pool}: All the other resources subcontract work
significantly amongst themselves.

Let's see if we can get any more granularity regarding the sub-groups in
``General Pool'' by inspecting the cases \& tasks executed by these
resources. Couldn't find an appropriate filter plug-in to do this in
ProM. Therefore, I switched to Disco and figured out that I need to
filter the event log to delete cases that the ``Singletons'' worked on
\& utilize the remaining cases to re-do this analysis.

\textbf{Approach I used}:

\begin{enumerate}
\def\labelenumi{\arabic{enumi}.}
\setcounter{enumi}{10}
\itemsep1pt\parskip0pt\parsep0pt
\item
  Click on ``Import'' icon in Disco.\\
\item
  Select the event log.\\
\item
  Click on ``Filter'' icon in the bottom left.\\
\item
  Click in the left pane to add filter.\\
\item
  Select ``Attribute - Removes events by attribute''.\\
\item
  Select ``Resource'' in the ``Filter by:'' option.\\
\item
  Select ``Mandatory'' in the ``Filtering mode:'' option.
\item
  Select ``Resource31'' only in ``Event values:'' pane.
\item
  Click on ``Apply filter'' button.
\end{enumerate}

\textbf{What I saw}:

\includegraphics{CoSeLog_Step07_Filter_Resource31.png}

2 cases in which all tasks were executed by Resource31.

\textbf{My analysis}:

The ``Singletons'' worked on 4 cases (10918 for Resource 42; 4328 for
Resource 36 \& {[}8012, 8014{]} for Resource 31) only. Moreover, in each
of these cases, all the tasks were conducted by one resource. Similarly,
I also found resources \{``test'' \& ``TEST''\} who each had one case
8061 \& 8047. I excluded these cases by applying a filter \& exported
the filtered event log.

Re-did tasks 1-8 as listed above for ``Step 07: inspect resource
utilization in ProM''. The clustering in ProM suggested expanding \&
splitting the ``Singletons'' group into ``Exclusive One-timers'' and
``Occasional Helpers:'' as listed further below.

After filtering the event log to delete cases that utilize these
resources, I re-did the social networking analysis in ProM.

\textbf{What I saw}:

\includegraphics{CoSeLoG_Step_07_excl_one-timers_SocialNet.png}

\textbf{My analysis}:

The resources may be grouped into the following categories:\\1.
\emph{Exclusive One-timers}: Resources \{31, 36, 42, test, TEST\} have
worked on their cases exclusively (Total 6 cases).

\begin{enumerate}
\def\labelenumi{\arabic{enumi}.}
\setcounter{enumi}{1}
\item
  \emph{Occasional Helpers}: Resources \{33, 35, 37, 38, 39, 40, 41, 43,
  admin3\}\\2.1 Resources \{40, 41, 43\} have helped with one activity
  on one case each.\\2.2 Resource37 has helped with 5 activities on one
  case.\\2.3 Resource39 has helped with one activity (T02) on 2
  cases.\\2.4 Resource38 has helped with activities on 2 cases.\\2.5
  Resource admin3 has helped with one activity on 3 cases.\\2.6
  Resource33 has helped with T07-* activities on 3 cases.\\2.7
  Resource35 has helped with multiple activities on 3 cases.

  Total 17 cases.
\item
  \emph{Isolated Couples}: Resources \{{[}13, 22{]}, {[}23, 25{]},
  {[}14, 34{]}, {[}28, 32{]}, {[}16, admin2{]}, {[}26, 30{]}, {[}21,
  29{]}, {[}06, 20{]}\} occasionally are outsourced / assign work. The
  cluster sub-groups are probably based on the tasks similarity.
\item
  \emph{Workgroup A}: Resources \{01, 05, 07, 11, 19\}
\item
  \emph{Workgroup B}: Resources \{{[}10, admin1{]}, {[}04, 24{]}, {[}02,
  03, 08, 09, 12, 15, 17, 18, 27{]}\}\\5.1 Resources 10 \& admin1 are
  out-scourced work by many resources but do not assign any work to
  others.\\5.2 Resources 04 \& 24 are out-sourced work by many resources
  and occassionally assign work to others.
\end{enumerate}

For, are all users executing activities from the start of the event log,
or are some users joining later? \& are users mainly executing
particular activities or are most users executing most of the
activities?

\textbf{Approach I used}:\\21. Click on ``Workspace'' icon.\\22. Select
event log excluding cases that involve \emph{Exclusive One-timers} or
\emph{Occasional Helpers}.\\23. Click on ``Visualize'' icon for the
selected resource.\\24. Select ``XDottedChart'' by clicking on ``Create
new\ldots{}'' drop-box.\\25. Select ``Dotted Chart'' tab.\\26. Select
the following options:\\ Component: `org:resource'\\ Time: Logical\\
Case order: Occurence of first event\\27. Click on ``Apply settings''
button.

\textbf{What I saw}:

\includegraphics{CoSeLoG_Step_07_Filtered_SocialDots.png}

The resources are displayed as rows and the transitions are organized
into columns based on when the transition was executed by that resource
for a certain case. The transitions are color coded.

\textbf{My analysis}:

Most of the resources are executing activities from the start of the
event log. It's hard to identify the exceptions. I wish the chart could
be interactive (e.g.~click on a dot and obtain more information).

Most resources are executing the first few activities \{TA, T02, T06\}.
The black color is assigned to multiple activities, so it's hard to tell
whether many resources are executing these activities or only a few.
Other activities appear to be executed by only a few.

\end{document}
